\documentclass[12pt,a4paper]{article}
\usepackage{amsmath,amssymb}
\usepackage{graphicx}
\usepackage{float}
\usepackage{tikz}
\usetikzlibrary{calc}
\usepackage[english]{babel}
\usepackage[utf8]{inputenc}
\usetikzlibrary{shapes, arrows, positioning}

\usepackage{amsmath,amssymb,mdframed}
\setcounter{page}{2}
\setlength{\oddsidemargin}{-0.25in}
\setlength{\textwidth}{6.5in}
\setlength{\topmargin}{-0.5in}
\setlength{\headsep}{1cm}
\setlength{\textheight}{9.0in}

\begin{document}
\begin{enumerate}

\renewcommand\labelenumi{\bfseries\theenumi.}

\item
Consider the following matrix:

\[
M =
\begin{pmatrix}
3a & a \\
2a & 2
\end{pmatrix},
\qquad a>0.
\]

\begin{enumerate}
  \item Show that for all $a>0$ the game defined by $(M, M^{\mathsf T})$ is \emph{not}
    a Prisoners' Dilemma.

  \item Find all \textbf{Nash equilibria} of the game as a function of $a$.

  \item Now consider a \textbf{two-type Moran process} with a population of size~$N$.
    Type~1 individuals play row~1; type~2 individuals play row~2.

    The fitness of each type is given by
    \[
    f_1(i) =
      \frac{(i-1) M_{11} + (N-i) M_{12}}{N-1},
    \qquad
    f_2(i) =
      \frac{i\, M_{21} + (N-i-1) M_{22}}{N-1},
    \]
    where $i$ is the number of type-1 individuals.

    Using the standard formula
    \[
    \rho_i
    =
    \frac{
      1 + \displaystyle\sum_{j=1}^{i-1}
            \prod_{k=1}^{j} \gamma_k
    }{
      1 + \displaystyle\sum_{j=1}^{N-1}
            \prod_{k=1}^{j} \gamma_k
    },
    \qquad
    \gamma_k = \frac{f_2(k)}{f_1(k)},
    \]

  \item Compute explicitly the fixation probability of a
    \textbf{single mutant} ($\rho_1$) for $N\in\{2, 3,4\}$.

  \item For $N\in\{2, 3,4\}$, analyse the fixation probability $\rho_1$ in the two limits
    \begin{enumerate}
      \item $a \to 0$,
      \item $a \to \infty$.
    \end{enumerate}

    Explain the evolutionary intuition behind these two limiting behaviours and
    relate them to the role of \textbf{fitness amplification} in
    frequency-dependent selection.
\end{enumerate}
\end{enumerate}
\end{document}
